\documentclass[a4paper,11pt]{article}

\usepackage[english]{babel}
\usepackage[utf8]{inputenc}
\usepackage{amsmath}
\usepackage{graphicx}
\usepackage[colorinlistoftodos]{todonotes}
\usepackage[lmargin=1.in,rmargin=1.in,tmargin=1.in,bmargin=1in]{geometry}
\usepackage{authblk}
\usepackage{setspace}

\title{The Gambler's Ruin Problem}
\author{Brian Weinstein - bmw2148}
\affil{Probability \& Statistics  (STAT-W4700), Columbia University, Fall 2015}
\date{December 15, 2015}

\begin{document}
\maketitle
\doublespacing



\section{Problem Statement}

\subsection{Setup}

In the Gambler's Ruin Problem, we have two gamblers, gambler $A$ and gambler $B$, who are playing a game against each other.

The total combined fortune of the two gamblers is $k$ dollars, with gambler $A$ starting with $i$ dollars, and gambler $B$ starting with $k-i$ dollars.

On each play of the game, gambler $A$ 




\end{document}