\documentclass[a4paper,11pt]{article}

\usepackage[english]{babel}
\usepackage[utf8]{inputenc}
\usepackage{amsmath}
\usepackage{graphicx}
\usepackage[colorinlistoftodos]{todonotes}
\usepackage[lmargin=1.in,rmargin=1.in,tmargin=1.in,bmargin=1in]{geometry}
\usepackage{authblk}
\usepackage{setspace}

\title{The Gambler's Ruin Problem}
\author{Brian Weinstein - bmw2148}
\affil{Probability \& Statistics  (STAT-W4700), Columbia University, Fall 2015}
\date{December 15, 2015}

\begin{document}
\maketitle
\doublespacing



\section{Problem Statement}

In the Gambler's Ruin Problem, we have a gambler $A$ and a casino $B$, who are playing a game against each other. The total combined fortune of the two is $k$ dollars, with the gambler starting with $i$ dollars, and the casino starting with $k-i$ dollars, where $i$ and $k-i$ are known positive integers. On each play of the game, the probability that $A$ will win one dollar from $B$ is $p$, where $0<p<1$, and the probability that $B$ will win one dollar from $A$ is $1-p$.

Suppose that the game is played repeatedly (and independently) until the fortune of either $A$ or $B$ is reduced to 0 dollars.



\section{Problem Solution}
Let $a_i$ denote the probability that gambler $A$ will reach $k$ dollars before it reaches 0 dollars, given that their initial fortune is $i$ dollars. Since each play of the game is independent of the others, the problem essentially starts over on each play, with the only difference being the ``initial'' fortunes of the gambler and casino.

The value of interest is $a_i$ for $i\in\{0,1,\ldots,k-1,k\}$. The cases of $i=0$ and $i=k$ are trivial. When $A$ runs out of money they can no longer play, and thus $a_0=0$, and when $A$ wins all $k$ dollars, the casino can no longer play, and thus $a_k=1$.








\end{document}